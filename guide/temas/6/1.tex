\textbf{Compilando}
El \textit{código fuente} no puede ser \textbf{interpretado} directamente por la computadora, primero debes convertirlo a instrucciones escritas en binario que se escriben en archivos llamados \textbf{ejecutables}. A este proceso se le llama \textbf{compilación}.

El \textit{ejecutable} de un \textit{código fuente} es generado por un \textbf{intérprete}. Cada lenguaje de programación tiene su propio \textit{intérprete}. Si ya haz instalado el (o los) lenguajes de programación que elegiste, ¡ya eres capaz de compilar tu primer programa!

\begin{tabular} {p{8.5cm} p{8.5cm}}
  \textbf{Windows} Los \textit{ejecutables} son aplicaciones .exe que puedes ejecutar haciéndoles doble clic desde el explorador, o escribiendo su nombre desde el \textbf{cmd}.

  Para fines prácticos, llamaremos al \textbf{cmd}: \textbf{Consola de Windows} &

  \textbf{Unix} Los \textit{ejecutables} generados por el \textit{intérprete}, son archivos con \textbf{permisos de ejecución}, y puedes ejecutarlos haciéndoles doble clic desde el explorador, o escribiendo ``./'' y el nombre del archivo desde la \textbf{consola}, así:

  \begin{lstlisting}
    ./nombre-del-archivo
  \end{lstlisting}

\end{tabular}

\textbf{Salida (Output)}

Cuando uno de estos programas es ejecutado, por lo general escribirá algo en la consola, y luego terminará. A ``eso'' que escribe en la \textit{consola}, se le llama \textbf{Output} (que significa ``Salida'' en inglés).


\textbf{Entrada (Input)}

Así como un programa puede darte respuestas mediante la salida, también puede pedirte respuestas. 

